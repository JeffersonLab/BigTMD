\documentclass[12pt]{article}
\usepackage{amsmath,amssymb,url,hyperref}
\usepackage[margin=0.4in,footskip=0.25in]{geometry}
\usepackage{graphicx} 
\usepackage{color}
\usepackage{enumerate}
\usepackage{slashed}
\usepackage{wasysym}
\newcommand{\bquote}[1]{\textcolor{blue}{ \begin{quote} ``#1" \end{quote}}}
\newcommand{\rquote}[1]{\textcolor{red}{ \begin{quote} #1 \end{quote}}}
\newcommand{\picineq}[1]{\ensuremath{\begin{array}{c} \includegraphics[scale=0.35]{#1} \end{array} } }
\newcommand{\question}[1]{\underline{\textcolor{red}{Question:}}{\; #1}}
\newcommand{\arrowcom}[1]{\textcolor{red}{ \\ \textbf{$\Longrightarrow$ #1} \\}}
\newcommand{\arrowcomtwo}[1]{\textcolor{magenta}{ \\ \textbf{$\Longrightarrow$ #1} \\}}
\newcommand{\three}[1]{{\bf #1}}
\hoffset=0.truein
\voffset=0truein
\hsize=6.5truein
\vsize=9truein
\parskip 5pt plus 3pt
\parindent 0pt
\def\page{\vfil\eject}
\newbox\tstrutbox
\setbox\tstrutbox=\hbox{\vrule height12.5pt depth4.5pt  width0pt}
\def\tstrut{\relax\ifmmode\copy\tstrutbox\else\unhcopy\tstrutbox\fi}
\newcommand{\no}{\nonumber \\}
\newcommand{\parz}[1]{\ensuremath{\left(#1\right)}}
\newcommand{\diff}[1]{\mathrm{d}#1}
\newcommand{\T}[1]{\boldsymbol{#1}_{\text{T}}}
\newcommand{\kT}{\ensuremath{k_{\rm T}}}
\newcommand{\ktmax}{\ensuremath{k_{\rm T max}}}
\newcommand{\ktmaxsq}{\ensuremath{k_{\rm T max}^2}}
\newcommand\3[1]{\boldsymbol{#1}}
\newcommand{\Tsc}[2]{#1_{#2\text{T}}}
\newcommand{\Tscsq}[2]{#1^2_{#2\text{T}}}
\usepackage{cancel}
\newcommand\Ccancel[2][black]{\renewcommand\CancelColor{\color{#1}}\cancel{#2}}
\newcommand{\xn}{\ensuremath{x_{\rm N}}}

\font\twelvess=cmss12
\font\twelvebold=cmbx12

%\twelvess
\begin{document}

\centerline{Quickly Produce SIDIS Transverse Momenta with BigkT}
\centerline{\today}
\vspace{.25in}
%%%%%%%%%%%%%%%%

This documentation explains how to generate the SIDIS cross section:
\begin{equation}
\frac{\diff{\sigma}{}}{\diff{x} \diff{Q^2} \diff{z} \diff{|\T{P}{}|} }
\end{equation}
in units of GeV$^{-5}$. Here, $\T{P}{}$ is the produced hadron's transverse momentum in the 
Breit frame and $x$, $z$, and $Q^2$ are the usual kinematical variables. 

\begin{itemize}
\item \underline{Installation and Example}:
\end{itemize}
\begin{enumerate}

\item Install Anaconda with python2 in your system which you can get for free at
     \rquote{https://www.anaconda.com}

\item Install lhapdf in your system. 

\item Open up a terminal. Below ``\$" denotes the ``terminal prompt"

\item You will need to make lhapdf reachable from python. For that you need 
      to set the environment variables ``PYTHONPATH" and
      ``LD\_LIBRARY\_PATH". For bash it can be
      done as
      \rquote{
      \$ export PATH=$<$path2lhapdf$\>$/bin:\$PATH \\
      \$ export PYTHONPATH=\$PYTHONPATH:$<$path2lhapdf$>$/lib/python2.7/site-packages/\\
      \$ export LD\_LIBRARY\_PATH=$<$lhapdf$>$/lib \\
      }
      Alternatively you can place these lines in your ``.bashrc" file

\item Clone the repository from github
     \rquote{\$ git clone git@github.com:JeffersonLab/BigTMD.git}

\item Go inside the repo directory
     \rquote{\$ cd BigTMD}

\item Copy the folders lhapdf/dsshpNLO  lhapdf/dsshmNLO inside to 
      \rquote{
      $<$path2lhapdf$>$/share/LHAPDF/
      }
      This will allow you to load DSS07 fragmentation functions from
      lhapdf 

\item Run the setup script (this takes some time)
      \rquote{\$ ./setup.py  }

\item The script ``sidis.py" orchestrates the full NLO calculation 
      for a given kinematic point 
        $$x,Q^2,z,q_{\rm T}=p_{\rm T}/z$$

\item Use ``driver.py" as an example. You can run it simply like 
      \rquote{\$ ./driver.py  }
\end{enumerate}

\begin{itemize}
\item \underline{Description}:
\end{itemize}

In the above, driver.py calls a function sidis.get\_xsec from sidis.py. In turn, sidis.py imports 
LO.py and the contents of $P_g$ and $P_{pp}$ in the NLO directory.  LO.py contains the leading order 
cross section directly, while $P_g$ and $P_{pp}$ contain all channels and charge configurations for the next-to-leading order 
 cross section. (See {\color{red} citation} for explanations of symbols, including $P_g$ and $P_{pp}$.) These functions are of the form 
 $$
 {\text{ fchn(channel)(charge configuration)}} .
 $$
Channels refers to the combination of incoming and outgoing parton identities:
\begin{align}
 &{}1 V: A \,  g  \to(q\to h) \, qb  &{} R: A \, g  &{}  \to(q\to h)  qb\, g  \\
 &{}2 V: A \,  q \to(q\to h) g   &{}R: A \, q  &{}\to(q\to h) \, g  \, g   \\
 &{}                                    &{} R: A \, q  &{} \to(q \to h) \, q' \, qb' \\
 &{} 3 V: A q  \to(g\to h) q   &{} R: A \, q  &{} \to(g \to h) \, q \, g   \\
 &{}4                                      &{} R: A \,  g  &{}\to(g \to h) \, q \, qb  \\
 &{}5                                      &{}R: A \, q  &{}\to(qb\to h)\, q \, qb  \\
 &{}6                                     &{} R: A \, q  &{}\to(q' \to h)\, q \, qb  \, ,
\end{align}
where $A$ is the virtual photon, $g$ is a gluon, $q$ and $q'$ denote different quark flavors, and $qb$ and $qb'$ are the antiparticles of $q$ and $q'$ respectively. 
$f \to h$ means it is the parton of flavor $f$ that hadronizes. $V$ and $R$ stand for ``real'' and ``virtual.'' So, for example, $3 R$ refers to graphs with two unobserved 
real emissions where a target quark leads to a hadronizing final state gluon and an unobserved quark and gluon. The correspondence between 
real and virtual graphs is in the sense of Table I of {\color{red} citation}.

By ``charge configuration'' we mean whether the photon couples directly to the charge of a target (anti)quark. A charge configuration $A$ is when the photon 
couples directly to the quark flavor in the pdf, configuration $C$ is when it does not, and $B$ is an interference between two such cases. See Fig.\ref{chgconf} for examples.
%---------------------
\begin{figure}[t]
\centering
\centering
  \begin{tabular}{c@{\hspace*{.01mm}}c@{\hspace*{.01mm}}c}
    \includegraphics[scale=0.5]{chgA}
    \hspace{0.5cm}
    &
    \hspace{0.5cm}
    \includegraphics[scale=0.5]{chgB}
    &
    \hspace{0.5cm}
    \includegraphics[scale=0.5]{chgC}
  \\
  (a) & (b) & (c)
  \end{tabular}
\caption{Examples of charge configurations $A$, $B$, and $C$. The solid black dot denotes the hadronizing parton.  
}
\label{chgconf}
\end{figure}
%---------------------

\end{document}
